% Options for packages loaded elsewhere
\PassOptionsToPackage{unicode}{hyperref}
\PassOptionsToPackage{hyphens}{url}
%
\documentclass[
  11pt,
  landscape]{article}
\usepackage{amsmath,amssymb}
\usepackage{lmodern}
\usepackage{ifxetex,ifluatex}
\ifnum 0\ifxetex 1\fi\ifluatex 1\fi=0 % if pdftex
  \usepackage[T1]{fontenc}
  \usepackage[utf8]{inputenc}
  \usepackage{textcomp} % provide euro and other symbols
\else % if luatex or xetex
  \usepackage{unicode-math}
  \defaultfontfeatures{Scale=MatchLowercase}
  \defaultfontfeatures[\rmfamily]{Ligatures=TeX,Scale=1}
\fi
% Use upquote if available, for straight quotes in verbatim environments
\IfFileExists{upquote.sty}{\usepackage{upquote}}{}
\IfFileExists{microtype.sty}{% use microtype if available
  \usepackage[]{microtype}
  \UseMicrotypeSet[protrusion]{basicmath} % disable protrusion for tt fonts
}{}
\makeatletter
\@ifundefined{KOMAClassName}{% if non-KOMA class
  \IfFileExists{parskip.sty}{%
    \usepackage{parskip}
  }{% else
    \setlength{\parindent}{0pt}
    \setlength{\parskip}{6pt plus 2pt minus 1pt}}
}{% if KOMA class
  \KOMAoptions{parskip=half}}
\makeatother
\usepackage{xcolor}
\IfFileExists{xurl.sty}{\usepackage{xurl}}{} % add URL line breaks if available
\IfFileExists{bookmark.sty}{\usepackage{bookmark}}{\usepackage{hyperref}}
\hypersetup{
  pdftitle={Lab 2: Land Use Inventory Table},
  hidelinks,
  pdfcreator={LaTeX via pandoc}}
\urlstyle{same} % disable monospaced font for URLs
\usepackage[margin=0.5in]{geometry}
\usepackage{graphicx}
\makeatletter
\def\maxwidth{\ifdim\Gin@nat@width>\linewidth\linewidth\else\Gin@nat@width\fi}
\def\maxheight{\ifdim\Gin@nat@height>\textheight\textheight\else\Gin@nat@height\fi}
\makeatother
% Scale images if necessary, so that they will not overflow the page
% margins by default, and it is still possible to overwrite the defaults
% using explicit options in \includegraphics[width, height, ...]{}
\setkeys{Gin}{width=\maxwidth,height=\maxheight,keepaspectratio}
% Set default figure placement to htbp
\makeatletter
\def\fps@figure{htbp}
\makeatother
\setlength{\emergencystretch}{3em} % prevent overfull lines
\providecommand{\tightlist}{%
  \setlength{\itemsep}{0pt}\setlength{\parskip}{0pt}}
\setcounter{secnumdepth}{-\maxdimen} % remove section numbering
\usepackage{setspace} \usepackage{titling} \setlength{\droptitle}{-10ex} \pretitle{\begin{flushleft}\Large\bfseries} \posttitle{\par\end{flushleft}} \preauthor{\begin{flushleft}\Large} \postauthor{\end{flushleft}} \predate{\begin{flushleft}} \postdate{\end{flushleft}}
\usepackage{booktabs}
\usepackage{longtable}
\usepackage{array}
\usepackage{multirow}
\usepackage{wrapfig}
\usepackage{float}
\usepackage{colortbl}
\usepackage{pdflscape}
\usepackage{tabu}
\usepackage{threeparttable}
\usepackage{threeparttablex}
\usepackage[normalem]{ulem}
\usepackage{makecell}
\usepackage{xcolor}
\ifluatex
  \usepackage{selnolig}  % disable illegal ligatures
\fi

\title{Lab 2: Land Use Inventory Table}
\author{}
\date{\vspace{-2.5em}}

\begin{document}
\maketitle

\vspace{-2.2cm}
\raggedright

Jennah Gosciak \newline February 7th, 2022 \newline Census Tracts
Representing Borough Park: 232, 234, 472, 474, and 476 \vspace{0.5cm}

\setlength{\tabcolsep}{6pt}
\renewcommand{\arraystretch}{1.2}

\begin{tabular}[t]{|>{\raggedright\arraybackslash}p{15em}|>{\raggedright\arraybackslash}p{7em}|>{\raggedright\arraybackslash}p{7em}|>{\raggedright\arraybackslash}p{7em}|>{\raggedright\arraybackslash}p{7em}|>{\raggedright\arraybackslash}p{7em}|>{\raggedright\arraybackslash}p{7em}|}
\hline
\textbf{Land Use} & \textbf{Number of Lots} & \textbf{Percentage of Total Number of Lots (\%)} & \textbf{Lot Area (sf)} & \textbf{Percentage of Total Lot Area (sf)} & \textbf{Building Floor Area (sf)} & \textbf{Percentage of Total Building Floor Area (sf)}\\
\hline
\textbf{Residential (One- \& Two-Family Buildings, Multifamily Walk-up Buildings, \& Multifamily Elevator Buildings)} & 3,563 & 81\% & 15,448,979 & 76\% & 15,568,517 & 65\%\\
\hline
\textbf{Mixed Residential \& Commerical Buildings} & 438 & 10\% & 1,616,356 & 8\% & 4,640,777 & 19\%\\
\hline
\textbf{Commercial \& Office Buildings} & 64 & 1\% & 447,160 & 2\% & 826,486 & 3\%\\
\hline
\textbf{Industrial \& Manufacturing Buildings} & 9 & 0\% & 82,267 & 0\% & 250,924 & 1\%\\
\hline
\textbf{Transportation \& Utility} & 18 & 0\% & 225,831 & 1\% & 25,600 & 0\%\\
\hline
\textbf{Public Facilities \& Institutions} & 159 & 4\% & 1,952,599 & 10\% & 2,600,506 & 11\%\\
\hline
\textbf{Open Space} & 6 & 0\% & 92,789 & 0\% & 400 & 0\%\\
\hline
\textbf{Parking Facilities} & 21 & 0\% & 77,149 & 0\% & 1,225 & 0\%\\
\hline
\textbf{Vacant Land} & 109 & 2\% & 344,809 & 2\% & 0 & 0\%\\
\hline
\textbf{Total} & 4,387 & 100\% & 20,287,939 & 100\% & 23,914,435 & 100\%\\
\hline
\end{tabular}

Data source: NYC Primary Land Use Tax Output (PLUTO) \newpage
\onehalfspacing The land use inventory table for Borough Park
(represented by five selected census tracts) shows that the majority of
all lots (81\%) are residential. The most common land use categories by
the number of lots are residential, mixed use, and public facilities and
institutions. This is accurate to my perception of the neighborhood.
Many buildings are either fully residential or have storefronts on the
ground floor with residential apartments above. There are few buildings
that are entirely commercial. Borough Park also has more than 300
religious institutions, one reason that there are so many lots under the
category of public facilities and institutions (Beyer, 2010). The land
use inventory table also shows a significant amount of vacant land (110
lots), which indicates areas that are either in the process of being
redeveloped or areas that may be underutilized. Given the need for
affordable housing, the Department of City Planning should evaluate
whether these areas can be sources of new housing (Beyer, 2010).
\newline \newline The land use inventory table suggests that most
buildings in Borough Park are low density. The ratio of total building
floor area to lot area is 1.18 (almost 1), which suggests that for a
number of buildings the lot area aligns with the building floor area.
This is especially true for residential buildings since the ratio of
building floor area to lot area is 1. For mixed use (residential and
commercial), the ratio is 2.87, which indicates that mixed use buildings
are higher density. Since vacant lots have a building floor area of 0,
it may be slightly misleading to compare the total building floor area
to the total lot area. For both lot area and building floor area,
residential buildings comprise a smaller share than they did when
analyzing land use by the number of lots. In contrast, public facilities
and institutions comprise a larger share. This finding suggests that
public facilities and institutions are likely larger building structures
compared to residential buildings. \newline \singlespacing Works Cited
\newline Beyer, G. (2010, Oct.~8). Borough Park, Brooklyn.
\textit{The New York Times.}
\url{https://www.nytimes.com/2010/10/10/realestate/10living.html}.

\end{document}
