% Options for packages loaded elsewhere
\PassOptionsToPackage{unicode}{hyperref}
\PassOptionsToPackage{hyphens}{url}
%
\documentclass[
  11pt,
  landscape]{article}
\usepackage{amsmath,amssymb}
\usepackage{lmodern}
\usepackage{ifxetex,ifluatex}
\ifnum 0\ifxetex 1\fi\ifluatex 1\fi=0 % if pdftex
  \usepackage[T1]{fontenc}
  \usepackage[utf8]{inputenc}
  \usepackage{textcomp} % provide euro and other symbols
\else % if luatex or xetex
  \usepackage{unicode-math}
  \defaultfontfeatures{Scale=MatchLowercase}
  \defaultfontfeatures[\rmfamily]{Ligatures=TeX,Scale=1}
\fi
% Use upquote if available, for straight quotes in verbatim environments
\IfFileExists{upquote.sty}{\usepackage{upquote}}{}
\IfFileExists{microtype.sty}{% use microtype if available
  \usepackage[]{microtype}
  \UseMicrotypeSet[protrusion]{basicmath} % disable protrusion for tt fonts
}{}
\makeatletter
\@ifundefined{KOMAClassName}{% if non-KOMA class
  \IfFileExists{parskip.sty}{%
    \usepackage{parskip}
  }{% else
    \setlength{\parindent}{0pt}
    \setlength{\parskip}{6pt plus 2pt minus 1pt}}
}{% if KOMA class
  \KOMAoptions{parskip=half}}
\makeatother
\usepackage{xcolor}
\IfFileExists{xurl.sty}{\usepackage{xurl}}{} % add URL line breaks if available
\IfFileExists{bookmark.sty}{\usepackage{bookmark}}{\usepackage{hyperref}}
\hypersetup{
  pdftitle={Lab 3: Zoning Analysis},
  hidelinks,
  pdfcreator={LaTeX via pandoc}}
\urlstyle{same} % disable monospaced font for URLs
\usepackage[margin=0.5in]{geometry}
\usepackage{graphicx}
\makeatletter
\def\maxwidth{\ifdim\Gin@nat@width>\linewidth\linewidth\else\Gin@nat@width\fi}
\def\maxheight{\ifdim\Gin@nat@height>\textheight\textheight\else\Gin@nat@height\fi}
\makeatother
% Scale images if necessary, so that they will not overflow the page
% margins by default, and it is still possible to overwrite the defaults
% using explicit options in \includegraphics[width, height, ...]{}
\setkeys{Gin}{width=\maxwidth,height=\maxheight,keepaspectratio}
% Set default figure placement to htbp
\makeatletter
\def\fps@figure{htbp}
\makeatother
\setlength{\emergencystretch}{3em} % prevent overfull lines
\providecommand{\tightlist}{%
  \setlength{\itemsep}{0pt}\setlength{\parskip}{0pt}}
\setcounter{secnumdepth}{-\maxdimen} % remove section numbering
\usepackage{setspace} \usepackage{titling} \setlength{\droptitle}{-10ex} \pretitle{\begin{flushleft}\Large\bfseries} \posttitle{\par\end{flushleft}} \preauthor{\begin{flushleft}\Large} \postauthor{\end{flushleft}} \predate{\begin{flushleft}} \postdate{\end{flushleft}}
\usepackage{booktabs}
\usepackage{longtable}
\usepackage{array}
\usepackage{multirow}
\usepackage{wrapfig}
\usepackage{float}
\usepackage{colortbl}
\usepackage{pdflscape}
\usepackage{tabu}
\usepackage{threeparttable}
\usepackage{threeparttablex}
\usepackage[normalem]{ulem}
\usepackage{makecell}
\usepackage{xcolor}
\ifluatex
  \usepackage{selnolig}  % disable illegal ligatures
\fi

\title{Lab 3: Zoning Analysis}
\author{}
\date{\vspace{-2.5em}}

\begin{document}
\maketitle

\vspace{-2.2cm}
\raggedright

Jennah Gosciak \newline February 14th, 2022 \newline Census Tracts
Representing Borough Park: 232, 234, 472, 474, and 476 \vspace{0.5cm}

\setlength{\tabcolsep}{6pt}
\renewcommand{\arraystretch}{1.2}

\textbf{Zoning Inventory Analysis for Borough Park (Census Tracts 232, 234, 472, 474, and 476)}

\begin{tabular}[t]{|>{\raggedright\arraybackslash}p{15em}|>{\raggedright\arraybackslash}p{7em}|>{\raggedright\arraybackslash}p{7em}|>{\raggedright\arraybackslash}p{7em}|>{\raggedright\arraybackslash}p{7em}|>{\raggedright\arraybackslash}p{7em}|>{\raggedright\arraybackslash}p{7em}|}
\hline
\textbf{Zoning Code} & \textbf{Number of Lots} & \textbf{Percentage of Total Number of Lots (\%)} & \textbf{Lot Area (sf)} & \textbf{Percentage of Total Lot Area (sf)} & \textbf{Building Floor Area (sf)} & \textbf{Percentage of Total Building Floor Area (sf)}\\
\hline
\textbf{R5} & 1,152 & 50\% & 3,734,330 & 53\% & 4,780,671 & 44\%\\
\hline
\textbf{R6} & 1,060 & 46\% & 2,948,009 & 42\% & 5,613,870 & 51\%\\
\hline
\textbf{C4-3} & 72 & 3\% & 196,487 & 3\% & 482,617 & 4\%\\
\hline
\textbf{M1-1} & 16 & 1\% & 69,434 & 1\% & 42,244 & 0\%\\
\hline
\textbf{C8-2} & 7 & 0\% & 57,517 & 1\% & 46,200 & 0\%\\
\hline
\textbf{R3-2} & 1 & 0\% & 1,850 & 0\% & 1,152 & 0\%\\
\hline
\textbf{Total} & 2,308 & 100\% & 7,007,627 & 100\% & 10,966,754 & 100\%\\
\hline
\end{tabular}

Data source: NYC Primary Land Use Tax Output (PLUTO) \newline \newline
The majority of lots in Borough Park are zoned residential, particularly
R5 and R6 zoning. R5 zoning produces lower density three- and
four-family homes; R6 can produce multi-family buildings on smaller lots
and denser, ``tower in the park'' developments on larger ones. The trend
is similar for lot area and building area; 95\% of the building area,
for example, is R5 and R6 zoning. A small percentage of lots is C4-3
zoning, which allows for commercial and residential uses. The C4-3 lots
reflect the presence of mixed use buildings with ground-floor
storefronts in the neighborhood. There are no special districts in
Borough Park. In general, the zoning inventory analysis aligns with my
perception of the neighborhood as primarily low- and medium-density,
with some mixed-use buildings on avenues and designated commercial
areas.

\newpage
\setlength{\tabcolsep}{6pt}
\renewcommand{\arraystretch}{1.2}

\textbf{FAR Utilization Analysis for Borough Park (Census Tracts 232, 234, 472, 474, and 476)}

\begin{tabular}[t]{|>{\raggedright\arraybackslash}p{15em}|>{\raggedright\arraybackslash}p{7em}|>{\raggedright\arraybackslash}p{7em}|>{\raggedright\arraybackslash}p{7em}|>{\raggedright\arraybackslash}p{7em}|>{\raggedright\arraybackslash}p{7em}|>{\raggedright\arraybackslash}p{7em}|}
\hline
\textbf{Zoning Code} & \textbf{Total Unutilized FAR (sqft)} & \textbf{Share of Total Unutilized FAR} & \textbf{Total Utilized FAR (sqft)} & \textbf{Share of Total Utilized FAR (sqft)} & \textbf{Total Max Floor Area Allowed (sqft)} & \textbf{Share of Total Max Floor Area Allowed}\\
\hline
\textbf{R6} & 3,492,044 & 63\% & 5,613,870 & 51\% & 7,163,662 & 56\%\\
\hline
\textbf{R5} & 1,465,670 & 26\% & 4,780,671 & 44\% & 4,667,912 & 37\%\\
\hline
\textbf{C4-3} & 458,994 & 8\% & 482,617 & 4\% & 668,056 & 5\%\\
\hline
\textbf{C8-2} & 96,154 & 2\% & 46,200 & 0\% & 115,034 & 1\%\\
\hline
\textbf{M1-1} & 46,363 & 1\% & 42,244 & 0\% & 69,434 & 1\%\\
\hline
\textbf{R3-2} & 0 & 0\% & 1,152 & 0\% & 925 & 0\%\\
\hline
\textbf{Total} & 5,559,224 & 100\% & 10,966,754 & 100\% & 12,685,023 & 100\%\\
\hline
\end{tabular}

Data source: NYC Primary Land Use Tax Output (PLUTO) \newline \newline
Overall, the primary use in Borough Park is residential followed by
mixed use (commercial and residential). There's around 5 million square
feet of unutilized FAR, primarily among medium density residential lots
and mixed use lots. There are a few lots (16) that are also zoned for
manufacturing. The most unutilized FAR involves lots with R6 zoning.
These lots could increase in density by more than 50\%. Similar to the
general patterns of land use in the neighborhood, R5 and C4-3 lots also
have significant unutilized FAR. However, total utilized area exceeds
the maximum allowable floor area for R5 lots. There are 571 lots overall
where the utilized area exceeds the maximum allowable area either
because the total building area is larger than the
\texttt{FAR*lot\ area} or because the residential area is only a subset
of the total building area. Many of these are lots with R5 or R6 zoning.
\newline \newline In absolute terms, R5 lots have a higher amount of
unutilized FAR, but C4-3 lots could increase their total utilized area
by almost 100\% if they were using the full FAR (i.e., the total
unutilized area is almost equal to the total utilized area). The same
zoning categories that have high amounts of unutilized FAR (R6, R5, and
C4-3) also have high amounts of total utilized FAR. Since R6, R5, and
C4-3 zoning types are the primary zoning types for the whole
neighborhood, any changes to the R6, R5, and C4-3 requirements would
have significant consequences. Efforts to modify the less dense R3-2
zoning requirements, which allow for one- and two-family homes, might be
ineffective in a neighborhood like Borough Park. Most of the buildings
are three-family homes or larger and already have unutilized FAR.
\newline \newline This table demonstrates that, purely based on zoning,
there is potential to increase density in Borough Park. However,
increasing density would require modifying or knocking down existing
structures, which is likely not desirable nor feasible. At the moment,
there does not seem to be a strong reason to modify the zoning in the
neighborhood; any new developments that make full use of the existing
FAR restrictions would increase density.

\end{document}
