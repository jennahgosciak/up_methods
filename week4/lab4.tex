% Options for packages loaded elsewhere
\PassOptionsToPackage{unicode}{hyperref}
\PassOptionsToPackage{hyphens}{url}
%
\documentclass[
  11pt,
  landscape]{article}
\usepackage{amsmath,amssymb}
\usepackage{lmodern}
\usepackage{ifxetex,ifluatex}
\ifnum 0\ifxetex 1\fi\ifluatex 1\fi=0 % if pdftex
  \usepackage[T1]{fontenc}
  \usepackage[utf8]{inputenc}
  \usepackage{textcomp} % provide euro and other symbols
\else % if luatex or xetex
  \usepackage{unicode-math}
  \defaultfontfeatures{Scale=MatchLowercase}
  \defaultfontfeatures[\rmfamily]{Ligatures=TeX,Scale=1}
\fi
% Use upquote if available, for straight quotes in verbatim environments
\IfFileExists{upquote.sty}{\usepackage{upquote}}{}
\IfFileExists{microtype.sty}{% use microtype if available
  \usepackage[]{microtype}
  \UseMicrotypeSet[protrusion]{basicmath} % disable protrusion for tt fonts
}{}
\makeatletter
\@ifundefined{KOMAClassName}{% if non-KOMA class
  \IfFileExists{parskip.sty}{%
    \usepackage{parskip}
  }{% else
    \setlength{\parindent}{0pt}
    \setlength{\parskip}{6pt plus 2pt minus 1pt}}
}{% if KOMA class
  \KOMAoptions{parskip=half}}
\makeatother
\usepackage{xcolor}
\IfFileExists{xurl.sty}{\usepackage{xurl}}{} % add URL line breaks if available
\IfFileExists{bookmark.sty}{\usepackage{bookmark}}{\usepackage{hyperref}}
\hypersetup{
  pdftitle={Lab 4: Zoning Compliance Analysis},
  hidelinks,
  pdfcreator={LaTeX via pandoc}}
\urlstyle{same} % disable monospaced font for URLs
\usepackage[margin=0.5in]{geometry}
\usepackage{graphicx}
\makeatletter
\def\maxwidth{\ifdim\Gin@nat@width>\linewidth\linewidth\else\Gin@nat@width\fi}
\def\maxheight{\ifdim\Gin@nat@height>\textheight\textheight\else\Gin@nat@height\fi}
\makeatother
% Scale images if necessary, so that they will not overflow the page
% margins by default, and it is still possible to overwrite the defaults
% using explicit options in \includegraphics[width, height, ...]{}
\setkeys{Gin}{width=\maxwidth,height=\maxheight,keepaspectratio}
% Set default figure placement to htbp
\makeatletter
\def\fps@figure{htbp}
\makeatother
\setlength{\emergencystretch}{3em} % prevent overfull lines
\providecommand{\tightlist}{%
  \setlength{\itemsep}{0pt}\setlength{\parskip}{0pt}}
\setcounter{secnumdepth}{-\maxdimen} % remove section numbering
\usepackage{setspace} \usepackage{titling} \setlength{\droptitle}{-10ex} \pretitle{\begin{flushleft}\Large\bfseries} \posttitle{\par\end{flushleft}} \preauthor{\begin{flushleft}\Large} \postauthor{\end{flushleft}} \predate{\begin{flushleft}} \postdate{\end{flushleft}} \usepackage{boldline}
\usepackage{booktabs}
\usepackage{longtable}
\usepackage{array}
\usepackage{multirow}
\usepackage{wrapfig}
\usepackage{float}
\usepackage{colortbl}
\usepackage{pdflscape}
\usepackage{tabu}
\usepackage{threeparttable}
\usepackage{threeparttablex}
\usepackage[normalem]{ulem}
\usepackage{makecell}
\usepackage{xcolor}
\ifluatex
  \usepackage{selnolig}  % disable illegal ligatures
\fi

\title{Lab 4: Zoning Compliance Analysis}
\author{}
\date{\vspace{-2.5em}}

\begin{document}
\maketitle

\vspace{-2.2cm}
\raggedright

Jennah Gosciak \newline February 21st, 2022 \newline

\onehalfspacing

The zoning compliance table for Borough Park (table 1) shows that in
general, with some deviations, the reality is consistent with what is
zoned on paper. The two residential districts (R5 and R6) all have
majority residential lots. According to the NYC Zoning and Land Use Map,
R5 districts are ``typically\ldots three-and four-story attached houses
and small apartment houses'' while R6 districts are ``built-up,
medium-density areas.'' R5 districts often act as a transition from low
to medium density. Still, there are some commercial, manufacturing, and
other uses in these districts. Lots in the R5 district are more in
alignment with the zoning code; 93\% of lots are residential. However,
the R6 district only has 88\% residential lots. 12\% of the lots are
commercial or other land uses. While this isn't a large share of lots,
it is a sizeable number (129 non-conforming lots) that suggests some
areas in Borough Park may benefit from changes to the zoning code, e.g.,
adding more commercial districts or commercial overlays.

The commercial and manufacturing districts (C4-3, C8-2, and M1-1), are
much smaller than the residential districts, but the land use patterns
do suggest that there could be better alignment. According to the NYC
Zoning and Land Use Map, C4-3 districts are ``regional commercial
centers\ldots that are located outside of the central business
districts.'' The uses typically include speciality and department
stores, theaters, and commercial office spaces. In Borough Park, the
C4-3 district, in particular, has a high share of residential lots
(85\%) and only 15\% of lots are actually used for commercial purposes.
C8-2 and M1-1 have a more varied mix of land use types, but they
comprise a small number of lots in total (together, only 23 lots). Given
that there are many commercial uses in residential areas, it might make
sense to expand the commercial zoning districts and add commercial
overlays to some of the residential districts while converting some of
the commercial districts to residential. This would improve the
alignment of real land use patterns and on-paper zoning districts. Since
the manufacturing district only has 16 lots, there may not be much value
in changing the zoning there. However, only three of the 16 lots are
actually used for manufacturing. Changing the zoning to commercial or
commercial and manufacturing mixed-use would be a small improvement.

In every zoning district in Borough Park, the majority of lots are
unutilized. This further suggests that revisions to the zoning code may
be necessary. C4-3 and R6 districts have the most unutilized lots: 99\%
and 92\% respectively. These are also districts where the land use
patterns most noticeably diverge from the zoning code. Given that
Borough Park has received little attention in citywide conversations
about rezoning, it might make sense to rezone some areas in the
neighborhood with particular focus on C4-3 and R6 districts. However,
lots in the neighborhood are overall in alignment with the zoning code
and there does not seem to be any need to urgently address these
inconsistencies.

\newpage

\textbf{Appendix} \newline \setlength{\tabcolsep}{6pt}
\renewcommand{\arraystretch}{1.2} Table 1. Zoning Compliance Table for
Borough Park (Census Tracts 232, 234, 472, 474, and 476)
\vspace*{-\baselineskip}

\begin{tabular}[t]{V{4}>{\raggedright\arraybackslash}p{7em}|>{\raggedright\arraybackslash}p{10em}|>{\raggedright\arraybackslash}p{5em}|>{\raggedright\arraybackslash}p{5em}|>{\raggedright\arraybackslash}p{5em}|>{\raggedright\arraybackslash}p{5em}|>{\raggedright\arraybackslash}p{5em}|>{\raggedright\arraybackslash}p{5em}V{4}}
\hlineB{4}
\multicolumn{1}{V{4}c|}{\textbf{ }} & \multicolumn{1}{c|}{\textbf{Zoning Code}} & \multicolumn{2}{c|}{\textbf{C4-3}} & \multicolumn{2}{c|}{\textbf{C8-2}} & \multicolumn{2}{cV{4}}{\textbf{M1-1}} \\
\cline{2-2} \cline{3-4} \cline{5-6} \cline{7-8}
\textbf{ } & \textbf{Category} & \textbf{Count of Lots} & \textbf{Share of Lots} & \textbf{Count of Lots} & \textbf{Share of Lots} & \textbf{Count of Lots} & \textbf{Share of Lots}\\
\hlineB{4}
 & Commercial & 11 & 15.3\% & 2 & 28.6\% & 3 & 18.8\%\\
\cline{2-8}
 & Manufacturing & 0 & 0\% & 3 & 42.9\% & 3 & 18.8\%\\
\cline{2-8}
 & Other & 0 & 0\% & 1 & 14.3\% & 3 & 18.8\%\\
\cline{2-8}
\multirow{-4}{7em}{\raggedright\arraybackslash \textbf{Land Use}} & Residential & 61 & 84.7\% & 1 & 14.3\% & 7 & 43.8\%\\
\cline{1-8}
 & Unutilized & 71 & 98.6\% & 6 & 85.7\% & 14 & 87.5\%\\
\cline{2-8}
\multirow{-2}{7em}{\raggedright\arraybackslash \textbf{Utilization}} & Utilized & 1 & 1.4\% & 1 & 14.3\% & 2 & 12.5\%\\
\cline{1-8}
\textbf{\textbf{}} & \cellcolor[HTML]{B4C6E7}{\textbf{Grand Total}} & \cellcolor[HTML]{B4C6E7}{\textbf{72}} & \cellcolor[HTML]{B4C6E7}{\textbf{100.0\%}} & \cellcolor[HTML]{B4C6E7}{\textbf{7}} & \cellcolor[HTML]{B4C6E7}{\textbf{100.0\%}} & \cellcolor[HTML]{B4C6E7}{\textbf{16}} & \cellcolor[HTML]{B4C6E7}{\textbf{100.0\%}}\\
\hlineB{4}
\end{tabular}

\begin{tabular}[t]{V{4}>{\raggedright\arraybackslash}p{7em}|>{\raggedright\arraybackslash}p{10em}|>{\raggedright\arraybackslash}p{5em}|>{\raggedright\arraybackslash}p{5em}|>{\raggedright\arraybackslash}p{5em}|>{\raggedright\arraybackslash}p{5em}V{4}}
\hlineB{4}
\multicolumn{1}{V{4}c|}{\textbf{ }} & \multicolumn{1}{c|}{\textbf{Zoning Code}} & \multicolumn{2}{c|}{\textbf{R5}} & \multicolumn{2}{c|}{\textbf{R6}} \\
\cline{2-2} \cline{3-4} \cline{5-6}
\textbf{ } & \textbf{Category} & \textbf{Count of Lots} & \textbf{Share of Lots} & \textbf{Count of Lots} & \textbf{Share of Lots}\\
\hlineB{4}
 & Commercial & 57 & 5\% & 98 & 9.3\%\\
\cline{2-6}
 & Manufacturing & 5 & 0.4\% & 4 & 0.4\%\\
\cline{2-6}
 & Other & 22 & 1.9\% & 27 & 2.5\%\\
\cline{2-6}
\multirow{-4}{7em}{\raggedright\arraybackslash \textbf{Land Use}} & Residential & 1,067 & 92.7\% & 930 & 87.8\%\\
\cline{1-6}
 & Unutilized & 871 & 75.7\% & 972 & 91.8\%\\
\cline{2-6}
\multirow{-2}{7em}{\raggedright\arraybackslash \textbf{Utilization}} & Utilized & 280 & 24.3\% & 87 & 8.2\%\\
\cline{1-6}
\textbf{\textbf{}} & \cellcolor[HTML]{B4C6E7}{\textbf{Grand Total}} & \cellcolor[HTML]{B4C6E7}{\textbf{1,152}} & \cellcolor[HTML]{B4C6E7}{\textbf{100.0\%}} & \cellcolor[HTML]{B4C6E7}{\textbf{1,060}} & \cellcolor[HTML]{B4C6E7}{\textbf{100.0\%}}\\
\hlineB{4}
\end{tabular}
\singlespacing

Data source: NYC Primary Land Use Tax Output (PLUTO) 21v3 \newline Table
note: last week's lab showed a single R3-2 lot in Borough Park. After
looking into that lot more extensively, I think that PLUTO misclassified
its census tract and community district. As a result, I removed it from
this analysis. \newline \newline \textbf{Works Cited} \newline New York
City's Zoning \& Land Use Map. (2021).
\url{https://zola.planning.nyc.gov/}.

\end{document}
